\documentclass{article}

\usepackage[margin=0.75in]{geometry}
\usepackage{qtree}
\qtreeprimes

\title{Syntax Trees in \LaTeX\ }
\date{}
\author{Rachael Steiner}

\begin{document}
\maketitle

This tutorial will show you how to make syntax trees using \LaTeX\ . It will assume you either know 
very basic \LaTeX\ (such as how to format a document) or can find out.

\section{Getting started}

To begin, you will need to use the package \textbf{qtree}. Include in your preamble:

\begin{verbatim}
	\usepackage{qtree}
\end{verbatim}

If you plan on using prime notation (e.g., for X-bar trees), also include the line:

\begin{verbatim}
	\qtreeprimes
\end{verbatim}

\section{The trees}

Start a tree with the command

\begin{verbatim}
	\Tree
\end{verbatim}

Trees are written in bracket notation. You can denote a node label by preceding 
it with a period. Any other text is taken to be the end of the branch.

\subsection{Example}

Take, for example, the phrase "the pretty tree":

\begin{verbatim}
	\Tree [.NP [.Det the ] [.N [.Adj pretty ] [.N tree ] ] ]
\end{verbatim}

\Tree [.NP [.Det the ] [.N [.Adj pretty ] [.N tree ] ] ]

\subsection{Formatting}



\end{document}
